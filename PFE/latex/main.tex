\documentclass[12pt]{report}

\usepackage{setspace}
\renewcommand{\baselinestretch}{1.5}
\usepackage{float}
\usepackage{lipsum}
\usepackage{multirow}
\usepackage{graphicx}
\usepackage{subcaption}
\usepackage[table,xcdraw,dvipsnames]{xcolor}
\usepackage{lscape}
\usepackage{titlesec}
\usepackage{titletoc}
\usepackage{titling}
\usepackage[T1]{fontenc}
\usepackage[utf8x]{inputenc}
\usepackage{amsmath}
\usepackage{caption}
\usepackage{adjustbox}
%\usepackage[a4paper,margin=0.5in]{geometry}
\usepackage[a4paper,%
            left=2cm,right=2cm,top=1cm,bottom=2cm,%
            footskip=-1cm]{geometry}
\usepackage[normalem]{ulem}
\usepackage[none]{hyphenat}
\usepackage[hidelinks]{hyperref}
\PassOptionsToPackage{hyphens}{url}\usepackage{hyperref}
\useunder{\uline}{\ul}{}

% No hyphenation (word wrapping) with text justification
\tolerance=1
\emergencystretch=\maxdimen
\hyphenpenalty=10000
\hbadness=10000

% Replace table caption name
\usepackage{caption}
% \captionsetup[table]{name=Table}

% For toc, lot and lof
\usepackage[titles]{tocloft}

% For liste of figures and tables!
\graphicspath{{./images/}}
\usepackage{array}

% Disable auto indentation globally
\setlength{\parindent}{0pt}

% Font options
\usepackage{mathptmx}
%\usepackage{tgtermes}
%\usepackage{tgpagella}
%\usepackage{helvet}
%\usepackage{tgbonum}

% fancy logo for page numbers!
\usepackage{fancyhdr,blindtext,tikz}
\usepackage{lastpage,refcount}

% Box frame for figures #\fbox{}
\setlength{\fboxsep}{0pt}
\setlength{\fboxrule}{0.69pt}

% Options: Sonny, Lenny, Glenn, Conny, Rejne, Bjarne, Bjornstrup
\usepackage[Conny]{fncychap}

%package for inserting raw text files
\usepackage{verbatim}
\usepackage{listings}

% Redifining from 0.1 to 1 ... 2 ... and so on
\renewcommand{\thesection}{\arabic{section}}

% Renaming Bibliography name
\renewcommand\bibname{References}

% Bibliography styling
\bibliographystyle{plain}

% Redifining from 1.1 to 1.a ... 1.b ... and so on
%\renewcommand{\thesubsection}{\alph{subsection}}

% Part rules
\titlecontents{part}[1em]
{\vskip 0.5ex}%
{}% numbered sections formatting
{\itshape\scshape}% unnumbered sections formatting
{}%

% Chapter rules
\titlecontents{chapter}[1em]
{\vskip 0.5ex}%
{}% numbered sections formatting
{\itshape\scshape}% unnumbered sections formatting
{\titlerule*[0.3pc]{.}\contentspage}%

% Section rules
\titlecontents{section}[1em]
{\vskip 0.2ex}%
{\hspace{0.15in}\contentslabel{0.13in}.\hspace{0.1in}}% numbered sections formatting
{\itshape}% unnumbered sections formatting
{\titlerule*[0.3pc]{.}\contentspage}%

% Subsection rules
\titlecontents{subsection}[3.5em]
{}%
{\contentslabel{0.10in}\hspace{0.25in}}% numbered sections formatting
{}% unnumbered sections formatting
{\titlerule*[0.3pc]{.}\contentspage}%

% Custom colors!
\definecolor{MYCOL}{HTML}{00318A}

% List of Figures settings:
\renewcommand{\listfigurename}{Liste Of Figures}
\renewcommand{\cftloftitlefont}{\vspace*{-0.6in}\hfill\color{Black}\fontsize{35}{43}\bfseries}
\renewcommand{\cftafterloftitle}{\hfill}
\setlength{\cftfigindent}{0pt} % left aligned Fig entries
\renewcommand{\cftfigpresnum}{Figure } % put before the number
\renewcommand{\cftfigaftersnum}{:} % put before the number
\addtolength{\cftfignumwidth}{2.5em} % extra space for \cftpresnum

% List of Tables settings:
\renewcommand{\listtablename}{Liste Of Tables}
\renewcommand{\cftlottitlefont}{\vspace*{-0.6in}\hfill\color{Black}\fontsize{35}{43}\bfseries}
\renewcommand{\cftafterlottitle}{\hfill}
\setlength{\cfttabindent}{0pt} % left aligned Tab entries
\renewcommand{\cfttabpresnum}{Table } % put before the number
\renewcommand{\cfttabaftersnum}{:} % put before the number
\addtolength{\cfttabnumwidth}{2.5em} % extra space for \cftpresnum

% Table of contents settings:
\renewcommand{\contentsname}{Table Des Matières}

\usepackage[framemethod=TikZ]{mdframed}

% My custom page style
\fancypagestyle{myfancy}{%   
   \fancyhf{}%
   \fancyfoot[C]{\tikz[baseline={(0,0)},anchor=center] \node[label={center:\thepage}]{\includegraphics[scale=.037]{../images/cloud.png}};}%
   \renewcommand{\headrulewidth}{0pt}%
   \renewcommand{\footrulewidth}{0pt}%
   \fancyhead{}
   \fancyfoot{}
   \fancyfoot[C]{\vspace{-0.16in}\textcolor{Gray}{\Large{Page}}\hspace{0.04in} \large{\thepage}\hspace{0.07in}\large{|}\hspace{0.07in}\large{\pageref{LastPage}}}
}%
\pagestyle{empty}

% Custom environment
\newenvironment{changemargin}[2]{%
\begin{list}{}{%
\setlength{\topsep}{0pt}%
\setlength{\leftmargin}{#1}%
\setlength{\rightmargin}{#2}%
\setlength{\listparindent}{\parindent}%
\setlength{\itemindent}{\parindent}%
\setlength{\parsep}{\parskip}%
}%
\item[]}{\end{list}}

% 'myfancy' page styling is also applied on chapters!
\usepackage{etoolbox}
\patchcmd{\chapter}{\thispagestyle{empty}}{\thispagestyle{myfancy}}{}{}

\titleformat{\chapter}
{\centering\fontsize{40}{50}\bfseries\color{Black}\scshape}
{}
{}
{\vspace{-1.7in}{\color{Black} \rule{\linewidth}{1.2mm} }}[\vspace{-0.35in}{\color{Black} \rule{\linewidth}{1.2mm} }\vspace{-1in}]

\titleformat{\section}
{\Large\bfseries\color{Black}}
{\thesection.\hspace{0.08in}}
{0em}
{}[\titlerule]

\titleformat{\subsection}
{\large\bfseries\color{Black}}
{\hspace{0.25in}\thesubsection\hspace{0.08in}}
{0em}
{}[]

%\titleformat{\subsubsection}
%{\bfseries\large}
%{\hspace{0.40in}}
%{0em}
%{}[]

\newenvironment{myindentpar}[1]%
  {\begin{list}{}%
          {\setlength{\leftmargin}{#1}}%
          \item[]%
  }
  {\end{list}}

\usepackage{tikz}
\usetikzlibrary{calc}

\usepackage{fullwidth}


\begin{document}

% For the box (theme)
\mdfdefinestyle{MyFrame}{%
    linecolor=black,
    outerlinewidth=2pt,
    roundcorner=0pt,
    innertopmargin=13pt,
    innerbottommargin=13pt,
    innerrightmargin=20pt,
    innerleftmargin=20pt,
    backgroundcolor=gray!10!white}

\begin{titlepage}

   \begin{tikzpicture}[remember picture, overlay]
      \draw[line width = 2pt] ($(current page.north west) + (0.25in,-0.25in)$) rectangle ($(current page.south east) + (-0.25in,0.25in)$);
   \end{tikzpicture}

   \begin{center}
       \vspace*{-0.4in}

       \begin{large}

           République Algérienne Démocratique et Populaire

           Ministère de l'Enseignement Supérieur et de la Recherche Scientifique

           Université M'Hamed Bougara de Boumerdès

       \end{large}

       \vspace{0.1in}

    \begin{figure}[h]
    \centering
        \includegraphics[width = 1.5in, height = 1in]{../images/UnivUMBB.jpg}
    \end{figure}
     \vspace*{-0.1in}
    \large{Faculté des Sciences}
    \\
    \vspace*{-0.1in}
    \large{Département d'informatique}
   \end{center}

    \hspace{0.2in}
    \large{\textbf{Domaine \,\; :}  Mathématiques Informatique}
    \kern 1in
    \large{\textbf{Année universitaire :}}

    \hspace{0.2in}
    \large{\textbf{Filière \quad\,\,\, :}  Informatique}
    \kern 2.65in
    2021 / 2022

    \hspace{0.2in}
    \large{\textbf{Spécialité \, :}  Technologie de l'information 
    % Ingénierie du Logiciel et Traitement de l'Information}

    \begin{center}
        \begin{large}

            \vspace{0.1in}

            \textit{\textbf{\uline{Projet fin d'étude}}}
        
        \end{large}

        \vspace{0.3in}

        \textit{\Huge{\textbf{\uline{Thème}}}}

        \vspace{0.2in}

        \begin{mdframed}[style=MyFrame]
            \begin{center}
            \color{Black}
              \Large{\textbf{La segmentation et Comptage des cellules biologiques}}

              \Large{\textbf{dans images de microscope par apprentissage profond}}
            \end{center}
        \end{mdframed}
    \end{center}

    \vspace{0.4in}

    \hspace{0.2in}
    \textit{\textbf{Présenté par :}}

    \hspace{0.2in}
    Gharbi Aghiles

    \hspace{0.2in}
    Neggazi Mohamed Lamine

\end{titlepage}


\newpage

\vspace*{0.2in}

\thispagestyle{empty}

\begin{center}
    {\color{Black} \rule{3in}{1.4mm} }\\
    \vspace{0.1in}
    \scshape{\fontsize{34}{46}{\bfseries{\color{Black}{Résumé}}}}
    \\
    \vspace{0.6in}
\end{center}
\cftaddtitleline{toc}{part}{\vspace{-0.12in}\color{Black}{Résumé}}{}
\begin{changemargin}{0.9cm}{0.9cm}
\hspace*{0.16in}
\end{changemargin}

Le comptage de cellules est une tâche fastidieuse qui bénéficierait grandement de l'automatisation. Le comptage précis des cellules CBC (Complete Blood Count) fournit des informations quantitatives essentielles et joue un rôle clé dans la recherche biologique ainsi que dans les applications industrielles et biomédicales. Malheureusement, la méthode de comptage manuel couramment utilisée demande beaucoup de temps, mal standardisée et n'est pas reproductible. La tâche est rendue encore plus difficile par le chevauchement des cellules. la mauvaise qualité de l'imagerie ... , nous comparons ici deux réseaux de neurones convolutionnels qui vont segmenter les globules globules (rouges, blancs et plaquettes) comme une première phase, et les compter dans la seconde phase en utilisant 3 algorithmes (Watershed, Connected Component Labeling et Circle Hough Transform).

\vspace{1in}

\begin{changemargin}{0.9cm}{0.9cm}
Mots clés : CNN, Comptage des cellules, segmentation d'image
\end{changemargin}

\newpage

\vspace*{0.2in}

\thispagestyle{empty}

\begin{center}
    {\color{Black} \rule{3in}{1.4mm} }\\
    \vspace{0.1in}
    \scshape{\fontsize{34}{46}{\bfseries{\color{Black}{Abstract}}}}
    \\
    \vspace{0.6in}
\end{center}
\cftaddtitleline{toc}{part}{\vspace{-0.12in}\color{Black}{Abstract}}{}
\begin{changemargin}{0.9cm}{0.9cm}
\hspace*{0.16in}

Cell counting is a tedious task that would benefit greatly from automation. Accurate cell counting CBC (Complete Blood Count) provides essential quantitative information and plays a key role in biological research as well as industrial and biomedical applications. Unfortunately, the commonly used manual counting method is time consuming, poorly standardized and not reproducible. The task is made even more difficult by overlapping cells and poor imaging quality. In this paper we compare between two convolutional neural networks that will segment the cells as a first phase, then count them in the second phase using 3 diffrent algorithms (Watershed, Connected Component Labeling et Circle Hough Transform).

\end{changemargin}

\vspace{1in}

\begin{changemargin}{0.9cm}{0.9cm}
Keywords: CNN, cell segmentation, cell counting, convolutional neural networks
\end{changemargin}

\newpage

\cftaddtitleline{toc}{part}{\vspace{-0.12in}\color{Black}{Table des matières}}{}
\tableofcontents

\newpage

\cftaddtitleline{toc}{part}{\vspace{-0.12in}\color{Black}{Liste des figures}}{}
\listoffigures

\newpage

\cftaddtitleline{toc}{part}{\vspace{-0.12in}\color{Black}{Liste des tableaux}}{}
\listoftables

\addtocontents{toc}{\protect\renewcommand{\protect\cftsecleader}{\protect\cftdotfill{\protect\cftsecdotsep}}}

\addtocontents{lof}{\protect\thispagestyle{empty}}
\addtocontents{lot}{\protect\thispagestyle{empty}}
\addtocontents{toc}{\protect\thispagestyle{empty}}

\newpage

\pagestyle{myfancy}

\vspace*{-0.2in}

\setcounter{page}{1}

\begin{center}
    {\color{Black} \rule{6.2in}{1.4mm} }\\
    \vspace{0.1in}
    \scshape{\fontsize{34}{46}{\bfseries{\color{Black}{General Introduction}}}}
    \\
    \vspace{0.5in}
\end{center}
\addcontentsline{toc}{chapter}{\vspace{-0.12in}\color{Black}{General Introduction}}
\hspace*{0.16in}

Blood carries out many vital functions as it circulates through the body. It transports oxygen from the lungs to other body tissues and carries away carbon dioxide. It carries nutrients from the digestive system to the cells of the body, and carries away wastes for excretion by the kidneys. Blood helps our body fight off infectious agents and inactivates toxins, stops bleeding through its clotting ability, and regulates our body temperature. Doctors rely on many blood tests to diagnose and monitor diseases. Some tests measure the components of blood itself; others examine substances found in the blood to identify abnormal functioning of various organs. Hence, we here propose a software system which will assist pathologists to detect blood cell count and help to find out the diseases. This information can be very helpful to a physician who, for example, is trying to identify the cause of a patient's diseases.

Earlier hematologists were performing microscopic Copied selected text to selection primary: examination and counting of blood cells manually, which  was very time-consuming and tedious process. Also, the accuracy of counting mainly depends on their expertise skill and their physical conditions, and because of cells complex nature, it still remains a challenging task to segment cells from its background and count them automatically.\\
Our work is to automate the task of cell counting, we will  try to find the best solution to preform the complete blood count (CBC), The solution is devided on two parts.
The first part is the segmentation where we need to segment the image to remove the noise and get a clear mask on which we are going preform the counting.In this phase we are comparing between two convolutional neural networks U-Net and SegNet.
The second part is the counting phase in which we will take  the output mask from the first phase and apply counting algorimths on it. We used 3 algorithms to count the cells Watershed, Connected Component Labeling, Circle Hough transform.

This thesis is presented in four chapters:\\

\textbf{Chapter 1 State of the art:}\\
\textbf{Chapter 2 Elaboration, Conception of multiple architecture and comparative study:}\\
\textbf{Chapter 3 Dataset collection:}\\
\textbf{Chapter 4 Implementation and experiments:}\\

\newpage

\vspace*{\fill}
\begin{center}
    {\color{Black} \rule{\linewidth}{1.2mm} }\\
\vspace{0.25in}
{\centering\fontsize{30}{40}{\bfseries{\color{Black}{\scshape{Chapter I : Blood cells \& artificial intelligence}}}}}
\vspace{0.35in}\\
    {\color{Black} \rule{\linewidth}{1.2mm} }
\end{center}
\vspace*{\fill}
\addcontentsline{toc}{chapter}{\color{Black}{Chapter I : Blood cells \& artificial intelligence}}
\setcounter{section}{0}

\newpage

\section{Introduction}
\vspace{0.2in}
\hspace*{0.16in}

\section{Related Work}
\vspace{0.2in}
\hspace*{0.16in}
Blood cell segmentation is the extraction of different blood cells from microscopic images. Blood cell counting is the process of counting the detected blood cells after segmentation. Many researchers have implemented methods for segmentation and counting of blood cells.\\

Kimbahune et al \textsuperscript{\cite{kimbahune2011blood}} have developed a method for segmenting and counting red blood cells (RBC) and white blood cells (WBC).
segmentation is done using Pulse-Coupled Neural Network (PCNN) and square tracing algorithm for countour tracing after de-noising it with PCNN combined with median filter, the counting is performed by scaning the image and using edge detection methods as square tracing algorithm. this method gave good results compared to state of art methods.\\
%this kimbahune article they didn't give any info about the database or the experimental results 

Bhavnani et al \textsuperscript{\cite{bhavnani2016segmentation}} have developed a method for segmenting and counting RBC (red blood cells), WBC (White blood cells) and platelets which is also called complete blood count (CBC), by using Otsu’s thresholding and morphological operations as a segmentation method, and for counting the are preforming a comparison between two methods: the watershed algorithm and Circular Hough Transform. The model takes an RGB image as an input apply some processing steps then uses Otsu's Thresholding to extract RBC and WBC separately with different threshold values then apply the two algorithm to compare the results, the model has no image size constraint because it's based only on image processing techniques and needs a small database to select the threshold values for RBC and WBC in this article they used 20 images. In the Experiment phase they used ALLIDB Database which contains 108 images with 1712x1368 and 2592x1944 resolution.The CHT method is the best in terms of accuracy with 92.67\% but it has some weaknesses with overlapping cells and morphological abnormalities. In the other side the watershed method which is a little bit adapted with overlapping and touching cells had an accuracy of 91.07\%.\\

Carlos X. Hern{\'{a}}ndez et al \textsuperscript{\cite{DBLP:journals/corr/abs-1802-10548}} have implemented a convolutional neural network (CNN) using a feature pyramid network (FPN) combined with a VGG style neural network for segmenting and counting of cells in a given microscopy image.
The dataset they used is BBBC005 \textsuperscript{\cite{ljosa2012annotated}} from Broad Institute's Bioimage Benchmark Collection, which consists of 9600 images and each image is 696x520 pixels but they were scaled down to 256x192 for the purposes of their experiment.

\newpage

Out of the total 9600 images only 600 of the images which have a corresponding mask were used for the FPN training. And 100 of those were used for fast prototyping and a standard of 80-20 train/test split for the final models.
On the other hand, the full 9600 images were used for the VGG network.
This approach achieved a relatively good accuracy of 95\% but with some failure cases such as:

\begin{itemize}
  \item High cell overlap
  \item Irregular cell shapes
  \item bad focal planes.
\end{itemize}

Tran, Thanh and Minh et al \textsuperscript{\cite{tran2019blood}} have developed a method for segmenting and counting RBC and WBCs by using the SegNet model initialised with weights from a pre-trained VGG-16 model, for the counting they first apply Distance transform with 4 different distance metrics, then they apply binary dilation. At the End, they apply the connected component labeling algorithm to count the number of separated cells in images mask. for the training they used 42 images from ALL-IDB1 \textsuperscript{\cite{labati2011all}} after they cropped them to decrease the computation time and
memory usage and reduce the number of RBC compared to WBC the result images have a resolution of 360*480*3 (RGB), they used 29 images for training and 13 for testing ,the model had a segmentation accuracy of 89\% and counting accuracy of 93.3\% on RBC and accuracy of 100\% on WBC with the testing database which has the cropped images of RBC and WBC.\\
On the first database with cropped images they had only few WBC but in this second database they have more WBC , the database 2 contains 108 only WBC images with the same size of database 1 360*480*3, they augmented the training dataset from 76 to 380 and used 32 images for testing, this second model focus only on the WBC which will increase the segmentation accuracy to 98.5\%, and have a counting accuracy of 97.29\%. The counting accuracy have decreased because of the clumped cells which is the weakness of this model.\\

K. Sudha and P. Geetha \textsuperscript{\cite{SUDHA2020639}} have developed a two stage framework which will segment the leukocytes (a type of WBC) with an edge strength-based Grabcut method as a first stage, in the second stage will count the cells using the novel gradient circular hough transform (GCHT) method. the model takes and RGB image convert it to HSV color space to extract the S component where the WBCs are more clear then applies the edge strength-based location detection the results are fed to fine segmentation using Grabcut Algorithm which will output the edge segmentation mask, for the counting the mask will be fed to the proposed GCHT Algorithm. In the experiments phase they used ALL-IDB \textsuperscript{\cite{labati2011all}} and Cellavision \textsuperscript{\cite{Zheng2018}} datasets, after resising the images to 256x256.\\
After the experiments the proposed method had rached an avreage segmentation accuracy of 99.32\% and a counting accuracy of 97.3\%. the new GCHT method can segment touched cells and even overlapped cells.\\

Yan Kong et al \textsuperscript{\cite{Kong:20}} have developed a two-stage framework using parallel modified U-Nets together with seed guided water-mesh algorithm for automatic segmentation and yeast cells counting which is used to observe the living conditions and survival of yeast cells under experimental conditions.

The cell images used in this study were captured by Olympus IX83 (Olympus Life Sciences, Tokyo, Japan) inverted microscope. They manually selected 20 raw DIC (differential interference contrast) images which contained a number of yeast cells and the annotations were done manually by laboratory technicians, they then obtain 40 images, 20 masked annotation images and the other 20 is center annotation images of yeast cells.

After spliting images into tiles of size 224x224 with a step stride of 65 and 33 pixels for the horizontal and vertical direction, respectively. They got 4360 raw image tiles and the corresponding center annotation and masked annotation images, from that set 3310 tiles were randomly selected as the training data set and rest was a validation set.

The raw test DIC images used in this study were sized approximately 1002x1998 pixels, but they were resized into 1092x2084 pixels so that each DIC image could be split into a grid of 8x16 image blocks. The image blocks were then fed into modified U-Net.

This method achieved a precision of over 96\% and an average recall rate of 99.35\%. however, there is a limitation using this approach, which is the detection of small objects.\\

Overton, Toyah and Tucker, Allan \textsuperscript{\cite{10.1007/978-3-030-44584-3_31}} have developed a method about segmentation and counting IDP (Internally Displaced people) and erythrocytes (red blood cells) using DO-U-Net (Dual Output U-Net) which outputs a segmentation mask and an edge mask then they substract them to get rid of the overlapping and the touching problem, the model trains on extremely small datasets (10 images) and gives a high segmentation accuracy, They selected 10 images of 108 from ALL-IDB dataset for training the model, the model takes images with a resolution of 188x188 and outputs a segmentation mask and edge mask of lower resolution 100x100, the experiments results have given an accuracy of 99.07 on a 5 randomly selected images from ALL-IDB, for the IDP they had 98.69\% for fixed resolution images and 94.66\% for scale-invariant satellite images.\\

\section{Conclusion}
\vspace{0.1in}
\hspace*{0.16in}


\newpage

\vspace*{\fill}
\begin{center}
    {\color{Black} \rule{\linewidth}{1.2mm} }\\
\vspace{0.25in}
 {\centering\fontsize{30}{40}{\bfseries{\color{Black}{\scshape{Chapter II : State of the art}}}}}
\vspace{0.35in}\\
    {\color{Black} \rule{\linewidth}{1.2mm} }
\end{center}
\vspace*{\fill}
\addcontentsline{toc}{chapter}{\color{Black}{Chapter II : State of the art}}
\setcounter{section}{0}

\newpage

\vspace*{\fill}
\begin{center}
    {\color{Black} \rule{\linewidth}{1.2mm} }\\
\vspace{0.25in}
 {\centering\fontsize{30}{40}{\bfseries{\color{Black}{\scshape{Chapter II : State of the art}}}}}
\vspace{0.35in}\\
    {\color{Black} \rule{\linewidth}{1.2mm} }
\end{center}
\vspace*{\fill}
\addcontentsline{toc}{chapter}{\color{Black}{Chapter II : State of the art}}
\setcounter{section}{0}

\newpage

\section{Introduction}
\vspace{0.2in}
\hspace{\parindent}
Blood cell segmentation is the extraction of different blood cells from microscopic images. Blood cell counting is the process of counting the detected blood cells after segmentation. Many researchers have implemented methods for segmentation and counting of blood cells using different approaches. In this section we devise the methods according to the segmentation approach.

\section{Related Work}

\subsection{Image Processing Approaches}
\hspace{\parindent}
Bhavnani et al. \textsuperscript{\cite{bhavnani2016segmentation}} have developed a method for segmenting and counting RBC (red blood cells), WBC (White blood cells) and platelets which is also called complete blood count (CBC), by using Otsu’s thresholding and morphological operations as a segmentation method, and for counting the are preforming a comparison between two methods: the watershed algorithm and Circular Hough Transform. The model takes an RGB image as an input apply some processing steps then uses Otsu's Thresholding to extract RBC and WBC separately with different threshold values then apply the two algorithm to compare the results, the model has no image size constraint because it's based only on image processing techniques and needs a small database to select the threshold values for RBC and WBC in this article they used 20 images. In the Experiment phase they used ALLIDB Database which contains 108 images with 1712x1368 and 2592x1944 resolution.The CHT method is the best in terms of accuracy with 92.67\% but it has some weaknesses with overlapping cells and morphological abnormalities. In the other side the watershed method which is a little bit adapted with overlapping and touching cells had an accuracy of 91.07\%.\

Guiliang, FENG et al. \textsuperscript{\cite{guiliang2016microscopic}} have a developed an algorithm that segments and counts cell images based on image definition, a Discrete Cosine Transform (DCT) is applied, which is proposed by N. Ahmed and Rao in 1974 \textsuperscript{\cite{ahmed1974discrete}}. Instead of the traditional watershed approach, the DCT method showed better results in comparison.\

However, there is a drawback to this approach, because this algorithm depends on image definition it relies on well focused images, consequently, when the images are out of focus the segmentation and counting is not reliable. But despite that drawback, it achieved a relatively high accuracy of over 90\% which is better that the watershed method.\\

K. Sudha and P. Geetha \textsuperscript{\cite{SUDHA2020639}} have developed a two stage framework which will segment the leukocytes (a type of WBC) with an edge strength-based Grabcut method as a first stage, in the second stage will count the cells using the novel gradient circular hough transform (GCHT) method. the model takes and RGB image convert it to HSV color space to extract the S component where the WBCs are more clear then applies the edge strength-based location detection the results are fed to fine segmentation using Grabcut Algorithm which will output the edge segmentation mask, for the counting the mask will be fed to the proposed GCHT Algorithm. In the experiments phase they used ALL-IDB \textsuperscript{\cite{labati2011all}} and Cellavision \textsuperscript{\cite{Zheng2018}} datasets, after resising the images to 256x256.\\
After the experiments the proposed method had rached an avreage segmentation accuracy of 99.32\% and a counting accuracy of 97.3\%.\\
The new GCHT method can segment touched cells and even overlapped cells.

\subsection{Machine Learning Approaches}
\hspace{\parindent}
Kimbahune et al. \textsuperscript{\cite{kimbahune2011blood}} have developed a method for segmenting and counting red blood cells (RBC) and white blood cells (WBC).
segmentation is done using Pulse-Coupled Neural Network (PCNN) and square tracing algorithm for countour tracing after de-noising it with PCNN combined with median filter, the counting is performed by scanning the image and using edge detection methods as square tracing algorithm. this method gave good results compared to state of art methods.\\
%this kimbahune article they didn't give any info about the database or the experimental results 

Carlos X. Hern{\'{a}}ndez et al. \textsuperscript{\cite{DBLP:journals/corr/abs-1802-10548}} have implemented a convolutional neural network (CNN) using a feature pyramid network (FPN) combined with a VGG style neural network for segmenting and counting of cells in a given microscopy image.\
The dataset they used is BBBC005 \textsuperscript{\cite{ljosa2012annotated}} from Broad Institute's Bioimage Benchmark Collection, which consists of 9600 images and each image is 696x520 pixels but they were scaled down to 256x192 for the purposes of their experiment.\

Out of the total 9600 images only 600 of the images which have a corresponding mask were used for the FPN training. And 100 of those were used for fast prototyping and a standard of 80-20 train/test split for the final models.\
On the other hand, the full 9600 images were used for the VGG network.
This approach achieved a relatively good accuracy of 81.75\% but with some failure cases such as:\
%the accuracy is calculated manually from the given results in the article
% 100 - (rmspe = np.sqrt(np.mean(np.square(((y_true - y_pred) / y_true)), axis=0)))

\begin{itemize}
  \item High cell overlap
  \item Irregular cell shapes
  \item bad focal planes.
\end{itemize}

Tran, Thanh and Minh et al. \textsuperscript{\cite{tran2019blood}} have developed a method for segmenting and counting RBC and WBCs by using the SegNet model initialised with weights from a pre-trained VGG-16 model, for the counting they first apply Distance transform with 4 different distance metrics, then they apply binary dilation. At the End, they apply the connected component labeling algorithm to count the number of separated cells in images mask. for the training they used 42 images from ALL-IDB1 \textsuperscript{\cite{labati2011all}} after they cropped them to decrease the computation time and memory usage and reduce the number of RBC compared to WBC the result images have a resolution of 360*480*3 (RGB), they used 29 images for training and 13 for testing ,the model had a segmentation accuracy of 89\% and counting accuracy of 93.3\% on RBC and accuracy of 100\% on WBC with the testing database which has the cropped images of RBC and WBC.\

On the first database with cropped images they had only few WBC but in this second database they have more WBC , the database 2 contains 108 only WBC images with the same size of database 1 360*480*3, they augmented the training dataset from 76 to 380 and used 32 images for testing, this second model focus only on the WBC which will increase the segmentation accuracy to 98.5\%, and have a counting accuracy of 97.29\%. The counting accuracy have decreased because of the clumped cells which is the weakness of this model.\\

Yan Kong et al. \textsuperscript{\cite{Kong:20}} have developed a two-stage framework using parallel modified U-Nets together with seed guided water-mesh algorithm for automatic segmentation and yeast cells counting which is used to observe the living conditions and survival of yeast cells under experimental conditions.\

The cell images used in this study were captured by Olympus IX83 (Olympus Life Sciences, Tokyo, Japan) inverted microscope. They manually selected 20 raw DIC (differential interference contrast) images which contained a number of yeast cells and the annotations were done manually by laboratory technicians, they then obtain 40 images, 20 masked annotation images and the other 20 is center annotation images of yeast cells.\

After spliting images into tiles of size 224x224 with a step stride of 65 and 33 pixels for the horizontal and vertical direction, respectively. They got 4360 raw image tiles and the corresponding center annotation and masked annotation images, from that set 3310 tiles were randomly selected as the training data set and rest was a validation set.

The raw test DIC images used in this study were sized approximately 1002x1998 pixels, but they were resized into 1092x2084 pixels so that each DIC image could be split into a grid of 8x16 image blocks. The image blocks were then fed into modified U-Net.

This method achieved a precision of over 99.74\% and an average recall rate of 99.35\%. however, there is a limitation using this approach, which is the detection of small objects.\\

Shahzad, Muhammad et al. \textsuperscript{\cite{shahzad2020robust}} have developed a custom convolutional encoder-decoder framework along with VGG-16 as the pixel-level feature extraction model to address the problem of whole-slide cell segmentation using the semantic segmentation approach. Their proposed framework works as follows: First, all the original images along with manually generated ground truth masks of each blood cell type are passed through the preprocessing stage. In the preprocessing stage, pixel-level labeling, RGB to grayscale conversion of masked image and pixel fusing, and unity mask generation are performed. After that, VGG16 is loaded into the system, which acts as a pretrained pixel-level feature extraction model. Finally, the training process is initiated on the proposed model.

They used ALL-IDB1 as their baseline dataset which consists of 108 whole-slide blood cell images, 59 (2592x1944) images were from healthy individuals and 49 (1712x1368) images from acute lymphoblastic leukemia (ALL) patients.

This approach achieved a classwise accuracies of 97.45\%, 93.34\%, and 85.11\% for RBCs, WBCs, and platelets, respectively, while global and mean accuracies remain 97.18\% and 91.96\%, respectively.\\

Overton, Toyah and Tucker, Allan \textsuperscript{\cite{10.1007/978-3-030-44584-3_31}} have developed a method which segments and counts IDP (Internally Displaced people) and erythrocytes (red blood cells) using DO-U-Net (Dual Output U-Net) which outputs a segmentation mask and an edge mask then they substract them to get rid of the overlapping and the touching problem, the model trains on extremely small datasets (10 images) and gives a high segmentation accuracy, They selected 10 images of 108 from ALL-IDB dataset for training the model, the model takes images with a resolution of 188x188 and outputs a segmentation mask and edge mask of lower resolution 100x100, the experiments results have given an accuracy of 98.31\% on a 5 randomly selected images from ALL-IDB, for the IDP they had 98.69\% for fixed resolution images and 94.66\% for scale-invariant satellite images.\\

Li, Dongming et al. \textsuperscript{\cite{li2021robust}} have developed a method for segmenting blood cells by combining neural ordinary differential equations (NODEs) with U-Net networks to improve the accuracy of image segmentation. In order to study the effect of ODE-solve on the speed and accuracy of the network, the ODE-block module was added to the nine convolutional layers in the U-Net network. Firstly, blood cell images are preprocessed to enhance the contrast between the regions to be segmented; secondly, the same dataset was used for the training set and testing set to test segmentation results. Then they select the location where the ordinary differential equation block (ODE-block) module is added, select the appropriate error tolerance, and balance the calculation time and the segmentation accuracy, in order to exert the best performance.\

Finally, the error tolerance of the ODE-block is adjusted to increase the network depth, and the training NODEs-UNet network model is used for cell image segmentation. 

The experiment dataset for this model was provided by the Center for Medical Image and Signal Processing (MISP) and the Department of Pathology, Isfahan University of Medical Sciences \textsuperscript{\cite{sarrafzadeh2014selection}}. MISP.rar contains 148 clear blood cell smear images with a size of 775x519 pixels. They picked up appropriate areas for convenient network training, then cropped 100 blood cell images with a size of 256x256 pixels by selecting a suitable area. To ensure the accuracy of the training model, they retained 20 images as the testing set and used the remaining 80 images to increase the dataset to 800 by data augmentation. Besides, they used a ratio of 3 : 1 as the training set and the validation set.

Using this approach to segment blood cell images in the testing set, it can achieve 95.3\% pixel accuracy and 90.61\% mean intersection over union. By comparing the U-Net and ResNet networks, the pixel accuracy of this network model is increased by 0.88\% and 0.46\%, respectively, and the mean intersection over union is increased by 2.18\% and 1.13\%, respectively.

\section{Comparative study}

\begin{table}[H]
\centering
\resizebox{\textwidth}{!}{%
\begin{tabular}{|
>{\columncolor[HTML]{FFFFFF}}l |
>{\columncolor[HTML]{FFFFFF}}c 
>{\columncolor[HTML]{FFFFFF}}c |
>{\columncolor[HTML]{FFFFFF}}c |
>{\columncolor[HTML]{FFFFFF}}c 
>{\columncolor[HTML]{FFFFFF}}c |
>{\columncolor[HTML]{FFFFFF}}c |
>{\columncolor[HTML]{FFFFFF}}c |}
\hline
\multicolumn{1}{|c|}{\cellcolor[HTML]{FFFFFF}\textbf{Reference}} &
  \multicolumn{1}{c|}{\cellcolor[HTML]{FFFFFF}\textbf{\begin{tabular}[c]{@{}c@{}}Segmentation\\ Approach\end{tabular}}} &
  \textbf{\begin{tabular}[c]{@{}c@{}}Counting\\ Approach\end{tabular}} &
  \textbf{\begin{tabular}[c]{@{}c@{}}Image\\ Size\end{tabular}} &
  \multicolumn{1}{c|}{\cellcolor[HTML]{FFFFFF}\textbf{\begin{tabular}[c]{@{}c@{}}Segmentation\\ Accuracy\end{tabular}}} &
  \textbf{\begin{tabular}[c]{@{}c@{}}Counting\\ Accuracy\end{tabular}} &
  \textbf{\begin{tabular}[c]{@{}c@{}}Database\\ Size\end{tabular}} &
  \textbf{\begin{tabular}[c]{@{}c@{}}Database\\ Name\end{tabular}} \\ \hline
\textbf{\begin{tabular}[c]{@{}l@{}}Kimbahune\\ et al\end{tabular}} &
  \multicolumn{1}{c|}{\cellcolor[HTML]{FFFFFF}PCNN} &
  Square tracing method &
  N/A &
  \multicolumn{1}{c|}{\cellcolor[HTML]{FFFFFF}N/A} &
  N/A &
  N/A &
  N/A \\ \hline
\textbf{\begin{tabular}[c]{@{}l@{}}Guiliang, FENG\\ et al.\end{tabular}} &
  \multicolumn{2}{c|}{\cellcolor[HTML]{FFFFFF}Discrete Cosine Transform (DCT)} &
  N/A &
  \multicolumn{2}{c|}{\cellcolor[HTML]{FFFFFF}90\%} &
  N/A &
  N/A \\ \hline
\textbf{\begin{tabular}[c]{@{}l@{}}Bhavnani\\ et al\end{tabular}} &
  \multicolumn{1}{c|}{\cellcolor[HTML]{FFFFFF}\begin{tabular}[c]{@{}c@{}}OTsu's thresholding \\ and morphological\\ operations\end{tabular}} &
  \begin{tabular}[c]{@{}c@{}}1-Circular Hough\\ Transfer\\ 2-Watershed Algorithm\end{tabular} &
  \begin{tabular}[c]{@{}c@{}}2594x1944\\ 1712x1368\end{tabular} &
  \multicolumn{2}{c|}{\cellcolor[HTML]{FFFFFF}\begin{tabular}[c]{@{}c@{}}1- 92.67\%\\ 2- 91.07\%\end{tabular}} &
  20 / 108 &
  ALLIDB \\ \hline
\textbf{\begin{tabular}[c]{@{}l@{}}Carlos\\ et al\end{tabular}} &
  \multicolumn{2}{c|}{\cellcolor[HTML]{FFFFFF}\begin{tabular}[c]{@{}c@{}}FPN combined with VGG style\\ neural network\end{tabular}} &
  256x196 &
  \multicolumn{1}{c|}{\cellcolor[HTML]{FFFFFF}95\%} &
   &
  80 / 20 &
  BBBC005 \\ \hline
\textbf{\begin{tabular}[c]{@{}l@{}}Tran, Thanh \\ and\\ Minh et al\end{tabular}} &
  \multicolumn{1}{c|}{\cellcolor[HTML]{FFFFFF}\begin{tabular}[c]{@{}c@{}}SegNet with weights\\ from a pre-trained\\ VGG-16\end{tabular}} &
  \begin{tabular}[c]{@{}c@{}}Distance Transform\\ and connected\\ component labeling\\ algorithm\end{tabular} &
  \begin{tabular}[c]{@{}c@{}}360x480x3\\ (RGB)\end{tabular} &
  \multicolumn{1}{c|}{\cellcolor[HTML]{FFFFFF}98.5\%} &
  97.29\% &
  380 / 32 &
  ALL-IDB1 \\ \hline
\textbf{\begin{tabular}[c]{@{}l@{}}K. Sudha and\\ P. Geetha\end{tabular}} &
  \multicolumn{1}{c|}{\cellcolor[HTML]{FFFFFF}\begin{tabular}[c]{@{}c@{}}Edge strength-based\\ Grabcut\end{tabular}} &
  \begin{tabular}[c]{@{}c@{}}Gradient Circular\\ Hough Transform\end{tabular} &
  256x256 &
  \multicolumn{1}{c|}{\cellcolor[HTML]{FFFFFF}99.32\%} &
  97.3\% &
   &
  ALL-IDB \\ \hline
\textbf{\begin{tabular}[c]{@{}l@{}}Yan Kong\\ et al\end{tabular}} &
  \multicolumn{1}{c|}{\cellcolor[HTML]{FFFFFF}\begin{tabular}[c]{@{}c@{}}Two parallel\\ modified U-Nets\end{tabular}} &
  \begin{tabular}[c]{@{}c@{}}Seed Guided Water-\\ Mesh Algorithm\end{tabular} &
  224x224 &
  \multicolumn{2}{c|}{\cellcolor[HTML]{FFFFFF}96\%} &
  3310 / 1050 &
  Self Annotated \\ \hline
\textbf{\begin{tabular}[c]{@{}l@{}}Overton, Toyah\\ and\\ Tucker, Allan\end{tabular}} &
  \multicolumn{1}{c|}{\cellcolor[HTML]{FFFFFF}DO-U-Net} &
  \begin{tabular}[c]{@{}c@{}}Marching Squares\\ Algorithm\end{tabular} &
  188x188 &
  \multicolumn{2}{c|}{\cellcolor[HTML]{FFFFFF}98.31\%} &
  10 / 5 &
  ALL-IDB1 \\ \hline
\end{tabular}%
}
\caption{Table that represents a comparative study of previous methods}
\label{Table 1}
\end{table}


\section{Conclusion}
\vspace{0.1in}
\hspace*{0.16in}


\newpage

\vspace*{\fill}
\begin{center}
    {\color{Black} \rule{\linewidth}{1.2mm} }\\
\vspace{0.25in}
    {\centering\fontsize{30}{40}{\bfseries{\color{Black}{\scshape{Chapter III : Dataset Collection}}}}}
\vspace{0.35in}\\
    {\color{Black} \rule{\linewidth}{1.2mm} }
\end{center}
\vspace*{\fill}
\addcontentsline{toc}{chapter}{\color{Black}{Chapter III : Dataset Collection}}
\setcounter{section}{0}

\newpage

\thispagestyle{empty}
\vspace*{\fill}
\begin{center}
    {\color{Black} \rule{\linewidth}{1.2mm} }\\
\vspace{0.25in}
{\centering\fontsize{30}{40}{\bfseries{\color{Black}{\scshape{Chapter III: Contribution}}}}}
\vspace{0.35in}\\
    {\color{Black} \rule{\linewidth}{1.2mm} }
\end{center}
\vspace*{\fill}
\addcontentsline{toc}{chapter}{\color{Black}{Chapter V : Contribution}}
\setcounter{section}{0}

\newpage

\section{Introduction}
\vspace{0.2in}
\hspace{\parindent}
As part of our research, we have treated the case of segmenting and counting Red, White blood cells and platelets which also known as CBC (Complete Blood Count), we are using the ALL-IDB\textsuperscript{\cite{pm77-2n23-20}} Dataset to train and evaluate our models.
In our case study, and from multiple articles, we can see that U-Net and Segnet models are dominating the field of cell segmentation and Medical Computer vision in general. we've chosen the article of Overton \textsuperscript{\cite{10.1007/978-3-030-44584-3_31}} because of the performance and optimisation of their segmentation model and we applied the same idea on the SegNet model, and for the counting methods we took the 3 of the most used methods to compare between them.

In this chapter, we test-out both U-Net and Segnet models, and analyse the results by comparing results of the two architectures.
We will also explore different machine learning algorithms for both pre-processing and post-processing.

\section{Proposed approach}
\vspace{0.2in}
\hspace{\parindent}
From all of the intel we have gathered, and previously read articles, all cell segmentation tasks (blood cell segmentation in particular) are mostly using U-Net and SegNet architectures for segmenting blood cell images.
We have used both the U-Net and SegNet models.
In the following sections, we will briefly analyze and compare both convolutional neural network (CNN) models with their perspective results.
And explain all the postprocessing methods we used for the counting of blood cells (red, white and platelets).

\section{DO-UNet}
\subsection{Definition}
\hspace{\parindent}
The U-Net is a convolutional neural network that was developed for biomedical image segmentation at the Computer Science Department of the University of Freiburg. The network is based on the fully convolutional network and its architecture was modified and extended to work with fewer training images and to yield more precise segmentations. In our case we are using DO-UNet from \textsuperscript{\cite{10.1007/978-3-030-44584-3_31}} which is a modified U-Net to produce dual outputs, which also known as contour aware network was first demonstrated by the DCAN architecture \textsuperscript{\cite{chen2016dcan}}. Based on a simple FCN, DCAN was trained to use the outer
contours of the areas of interest to guide the training of the segmentation masks. This led to improved semantic and instance segmentation of the model, which in their case, looked at non-overlapping features in biomedical imaging.
With the aim of counting closely co-located and overlapping cells, we are predominantly interested in the correct detection of individual objects as
opposed to the exact precision of the segmentation mask itself. An examination
of the hidden convolutional layers of the classical U-Net showed that the penultimate layer of the network extracts information about the edges of the cells, so the idea is to output the cell mask + edge mask then do a substraction to break the overlapping cells.

\subsection{Architecture}
\hspace{\parindent}
They Started with the classical U-Net then reduced the number of
convolutional layers and skip connections in the model. Simultaneously, they minimised the complexity of the model by looking at smaller input regions of the images, thus minimising the memory footprint of the model. They follow the approach of Ronneberger et. al. \textsuperscript{\cite{10.1007/978-3-030-44584-3_31}} by using unpadded convolutions throughout the network, resulting in a model with smaller output edge and mask (100 × 100 px) corresponding to a central region of a larger (188 × 188 px) input image region. DO-U-Net uses two, independently trained, output layers of identical size. Figure \ref{fig:DO-UNET} shows the DO-U-Net architecture.

\begin{figure}[H]
\centering
  \vspace{0.2in}
    \centerline{\includegraphics[width = \linewidth]{../images/DO-UNET.png}}
    \caption{DO-UNet architecture}
    \label{fig:DO-UNET}
\end{figure}

\newpage

\section{Segnet}
\subsection{Definition}
\hspace{\parindent}
The SegNet neural network, developed by Alex Kendall, Vijay Badrinarayanan, and Roberto Cipolla, all from the University of Cambridge, is a convolutional neural network used for semantic pixel wise labeling. This problem is more commonly called semantic segmentation. \textsuperscript{\cite{badrinarayanan2017segnet}}

\subsection{Architecture}
\hspace{\parindent}
SegNet has an encoder network and a corresponding decoder network, followed by a final pixelwise classification layer. This architecture is illustrated in the figure below.
The model we used has the 13 encoder layers obtained from the VGG16 network, and 13 decoder layers to match the same number of encoder layers. The final decoder output is fed to a multi-class soft-max classifier to produce class probabilities for each pixel independently (pixelwise).

Each encoder in the encoder network performs convolution with a filter bank to produce a set of feature maps. These are then batch normalized. Then an element-wise rectified- linear non-linearity (ReLU) max(0, x) is applied. Following that, max-pooling with a 2x2 window and stride 2 (non-overlapping window) is performed and the resulting output is sub-sampled by a factor of 2. Max-pooling is used to achieve translation invariance over small spatial shifts in the input image.

The appropriate decoder in the decoder network upsamples its input feature map(s) using the memorized max-pooling indices from the corresponding encoder feature map(s). This step pro- duces sparse feature map(s). This SegNet decoding technique is illustrated in the below figure.
These feature maps are then convolved with a trainable decoder filter bank to produce dense feature maps. A batch normalization step is then applied to each of these maps. Note that the decoder corresponding to the first encoder (closest to the input image) produces a multi-channel feature map, although its encoder input has 3 channels (RGB).
This is unlike the other decoders in the network which produces feature maps with the same number of size and channels as their encoder inputs. The high dimensional feature representation at the output of the final decoder is fed to a trainable soft-max classifier. \textsuperscript{\cite{badrinarayanan2017segnet}}\

The input shape we used is (128x128x3) and the output shape is (128x128), there is no loss in resolution because segnet uses the option `same' padding on each encoder layer.
We also tested 3 different loss functions (tversky loss, binary crossentropy and Mean Squared Error MSE), out of which the MSE outperformed the others.

\vspace{0.1in}

\begin{figure}[H]
\centering
  \vspace{-0.1in}
    \centerline{\includegraphics[width = 3.5in]{../images/segnet.png}}
    \caption{SegNet architecture}
\end{figure}

\section{Dataset}
\hspace{\parindent}
For both models, we decided to work with the updated ALL-IDB1 dataset which has 13 RBC edges and masks, 108 WBC, Platelets Masks and 13 RBC images which has the count information, but the WBC and Platelets don't have the count information where we had to use manual count and algorithms to find the count information for these images.\

10 images with their perspective masks and edge masks were chosen for red blood cell training and 3 as a test dataset, for white blood cells 73 images with their masks, and 33 as a test dataset.
For platelets, we used 71 for training and 31 as a test dataset.
Only red blood cells have edge masks, because we need to get rid of overlapped cells, white blood cells and platelets dont need the edge masks, using masks only can retrieve all the necessary features, because both white blood cells and platelets rarely overlap.
The images will be sliced to the input size of the according model to match the input shape of the models.
The resulting train dataset will be 3916 image, mask, and edge tiles (a total of 11748 tiles).
For the test dataset 1072 image, mask, and edge tiles (a total of 3216 tiles) for red blood cells.
As for white blood cells, 28126 image and mask tiles were used for training (a total of 56252 tiles), and 15892 image and mask tiles were used as a test dataset for white blood cells (a total of 31784 tiles).
Finally, for platelets we used 27650 image and mask tiles were used for training (a total of 55300), and 14410 image and mask tiles as a test dataset for platelets (a total of 28820).

\vspace{0.1in}

\begin{table}[H]
\centering
\resizebox{\textwidth}{!}{%
\begin{tabular}{|
>{\columncolor[HTML]{FFFFFF}}c 
>{\columncolor[HTML]{FFFFFF}}c |
>{\columncolor[HTML]{FFFFFF}}c |
>{\columncolor[HTML]{FFFFFF}}c |c|c|c|c|c|c|}
\hline
\multicolumn{2}{|c|}{\cellcolor[HTML]{FFFFFF}\textbf{Dataset}}                                              & \textbf{\begin{tabular}[c]{@{}c@{}}Train\\ images\end{tabular}} & \textbf{\begin{tabular}[c]{@{}c@{}}Test\\ images\end{tabular}} & \textbf{\begin{tabular}[c]{@{}c@{}}Train\\ Tiles\end{tabular}} & \textbf{\begin{tabular}[c]{@{}c@{}}Test\\ Tiles\end{tabular}} & \textbf{\begin{tabular}[c]{@{}c@{}}Total\\ train\\ images\end{tabular}} & \textbf{\begin{tabular}[c]{@{}c@{}}Total\\ test\\ images\end{tabular}} & \textbf{\begin{tabular}[c]{@{}c@{}}Total\\ train\\ tiles\end{tabular}} & \textbf{\begin{tabular}[c]{@{}c@{}}Total\\ test\\ tiles\end{tabular}} \\ \hline
\multicolumn{1}{|c|}{\cellcolor[HTML]{FFFFFF}}                                             & \textbf{Image} & 10                                                              & 3                                                              & 3916                                                           & 1072                                                          &                                                                         &                                                                        &                                                                        &                                                                       \\ \cline{2-6}
\multicolumn{1}{|c|}{\cellcolor[HTML]{FFFFFF}}                                             & \textbf{Mask}  & 10                                                              & 3                                                              & 3916                                                           & 1072                                                          &                                                                         &                                                                        &                                                                        &                                                                       \\ \cline{2-6}
\multicolumn{1}{|c|}{\multirow{-3}{*}{\cellcolor[HTML]{FFFFFF}\textbf{Red Blood Cells}}}   & \textbf{Edge}  & 10                                                              & 3                                                              & 3916                                                           & 1072                                                          & \multirow{-3}{*}{30}                                                    & \multirow{-3}{*}{9}                                                    & \multirow{-3}{*}{\textbf{11748}}                                       & \multirow{-3}{*}{\textbf{3216}}                                       \\ \hline
\multicolumn{1}{|c|}{\cellcolor[HTML]{FFFFFF}}                                             & \textbf{Image} & 73                                                              & 33                                                             & 28126                                                          & 15892                                                         &                                                                         &                                                                        &                                                                        &                                                                       \\ \cline{2-6}
\multicolumn{1}{|c|}{\multirow{-2}{*}{\cellcolor[HTML]{FFFFFF}\textbf{White Blood Cells}}} & \textbf{Mask}  & 73                                                              & 33                                                             & 28126                                                          & 15892                                                         & \multirow{-2}{*}{146}                                                   & \multirow{-2}{*}{66}                                                   & \multirow{-2}{*}{\textbf{56252}}                                       & \multirow{-2}{*}{\textbf{31784}}                                      \\ \hline
\multicolumn{1}{|c|}{\cellcolor[HTML]{FFFFFF}}                                             & \textbf{Image} & 71                                                              & 31                                                             & 27650                                                          & 14410                                                         &                                                                         &                                                                        &                                                                        &                                                                       \\ \cline{2-6}
\multicolumn{1}{|c|}{\multirow{-2}{*}{\cellcolor[HTML]{FFFFFF}\textbf{Platelets}}}         & \textbf{Mask}  & 71                                                              & 31                                                             & 27650                                                          & 14410                                                         & \multirow{-2}{*}{142}                                                   & \multirow{-2}{*}{62}                                                   & \multirow{-2}{*}{\textbf{55300}}                                       & \multirow{-2}{*}{\textbf{28820}}                                      \\ \hline
\end{tabular}%
}
\caption{Dataset used for segnet}
\label{Dataset used for segnet}
\end{table}


\section{Dataset augmentation}
\hspace{\parindent}
We used the same dataset augmention on all cells (red, white blood cells, and platelets) in both models UNet and SegNet.
The augmentation we used was custom which involves the following steps:
\begin{enumerate}
    \item Pick a random image from the train dataset.
    \item Get the x and y coordinates randomly from the chosen image.
    \item Rescale the image randomly to a smaller size then scale it back to the original size to reduce quality.
    \item Take a slice of the image and mask accordingly and also edge if available.
    \item Skip the image if it dosn't contain out object of interest
    \item Resize the image and mask to the model input.
    \item Randomly rotate and flip the image chip.
    \item Randomly augment the colors (luminosity and saturation).
\end{enumerate}

\begin{figure}[H]
\centering
  \vspace{-0.1in}
    \centerline{\includegraphics[width = 7in]{../images/Diag_RBC_DOUNET_SegNet.png}}
    \caption{Schema of the segmentation and counting steps of the RBC's}
    \label{fig:scheme_RBC}
\end{figure}

\begin{figure}[H]
\centering
  \vspace{-0.1in}
    \centerline{\includegraphics[width = 7in]{../images/Diag_WBC_PLT_UNET_SegNet.png}}
    \caption{Schema of the segmentation and counting steps of the WBC's and Platelets}
    \label{fig:scheme_WBC_PLT}
\end{figure}

\section{Counting}
\vspace{0.2in}
\hspace{\parindent}
After having segmented blood cell images (red, white blood cells and platelets), we use multiple post-processing methods (machine learning algorithms) to get the coordinates of circles and count them. as we can see in fig \ref{fig:scheme_RBC} and \ref{fig:scheme_WBC_PLT}.\\
We can see below are all the algorithms we used to get an relatively accurate blood cell count:

\subsection{Circle Hough Transform}
\hspace{\parindent}
Circle Hough Transform (CHT) is machine learning algorithm used to extract features (circles) from imperfect images.
We modified its parameters (Minimum distance, Minimum and Maximum radius...) for each type of blood cells (red, white and platelets).\\
Note that we do not rely on this approach to count white blood cells, because most white blood cells have different shapes.
Therefore, this method is useless when it comes to white blood cells counting.

We Modified this method by adding a loss function which will help us to eliminate False Positives circles by calculating the percentage of the intersection between the circle and the cell mask. we improved the counting accuracy by more than 20\% with a threshold intersection percentage of 60\%.\\
As we can see in fig \ref{fig:cht_scheme} the steps of the counting with the CHT method:
\begin{itemize}
    \item we first take the mask/edge from the model.
    \item we apply a threshold on the mask to binarize it.
    \item we apply our surface filter algorithm to filter object that are not in the size range of the cell that we are counting.
    \item apply the circle hough transform to detect the circles in the cleaned image.
    \item feed the binary mask and the obtained circles from CHT to calculate loss of each circle (percentage of intersection).
    \item return the final number of circles that meet the threshold condition which is the circle count.

  \end{itemize} 
\begin{figure}[H]
\centering
  \vspace{-0.1in}
    \centerline{\includegraphics[width = 7in]{../images/CHT_scheme.png}}
    \caption{CHT schema applied to count blood cells}
    \label{fig:cht_scheme}
\end{figure}

\subsection{Connected Component Labeling}
\hspace{\parindent}
Connected Component Labeling (CCL) is machine learning algorithm used to detect connected regions in a binary image.
Before applying the connected component labeling, we convert the images to gray-scale. Then, a binary threshold is applied to the images to get binary values.
Finally, we apply the connected component labeling to get the labels and map component labels to the resulting image, and the number of labels is the cell count accordingly.

\begin{figure}[H]
\centering
\begin{minipage}{.5\textwidth}
  \centering
  \centerline{\includegraphics[width = 77mm]{../images/Im0001_1_substraction.jpg}}
    \subcaption{input Image}
\end{minipage}%
\begin{minipage}{.5\textwidth}
  \centering
  \centerline{\includegraphics[width = 77mm]{../images/Im0001_1_connected_compounent_labeling.jpg}}
    \subcaption{output image}
\end{minipage}
  \caption{Example of connected component labeling}
\end{figure}

\subsection{Watershed}
\hspace{\parindent}
Watershed algorithms (also called drainage divide) are used in image processing primarily for object segmentation purposes, that is, for separating different objects in an image. the main purpose of using watershed in this phase is to segment the touching and overlapping cells, the watershed takes two inputs, first i takes an image with different intensity levels in our-case the distance transform of our mask where the intensity levels represents reliefs. the second input is the water sources in our-case we extracted local maxima from the distance transform image. we can see below the steps we used to count the cells.

\begin{enumerate}
    \item \textbf{Compute the Euclidean distance}: we compute euclidean distance from every binary pixel to the nearest zero pixel, this map will be used as our relief map in the watershed algorithm.
    \item \textbf{We find peaks in the distance map}: we search for peaks in our euclidean distance map which is the local maxima in each region, which are the highest points in the map (higher intensity levels), which we will use as water sources in the watershed algorithm.
    \item \textbf{Apply connected component labeling on the peak map}: we apply CCL algorithm which is also called 8-connectivity algorithm to label the peaks (label each water source).
    \item \textbf{Apply the Watershed algorithm on the reversed distance map using the labeled peaks}: at the end we feed the reversed distance map and the water sources map (local maxima) to the watershed algorithm to get the segmented image. 
\end{enumerate}

Here is a schema presenting the previously mentioned steps:

\begin{figure}[H]
\centering
  \vspace{-0.1in}
    \centerline{\includegraphics[width = 7in]{../images/watershed.png}}
    \caption{Watershed schema applied to count white blood cells}
\end{figure}

\section{Metrics And Loss Functions}
\hspace{\parindent}
Loss functions are one of the important ingredients in deep learning-based medical image segmentation methods. In the past four years, more than 20 loss functions have been proposed for various segmentation tasks. Most of them can be used in any segmentation tasks in a plug-and-play way, we can see in (fig \ref{fig:LossFunctions}) the relations between the most used Loss Functions, we will present below the used Loss Functions in our paper.

\begin{figure}[H]
\centering
  \vspace{-0.1in}
    \centerline{\includegraphics[width = \linewidth]{../images/LossFunctions.png}}
    \caption{Loss Functions}
    \label{fig:LossFunctions}
\end{figure}

In this section, we will discuss the loss functions and metrics that we used to train and evaluate our models.

When performing classification predictions (pixel-wise classification in our case) there's four types of outcomes that could occur.

\begin{enumerate}
    \item \textbf{True positives} are when you predict an observation belongs to a class and it actually does belong to that class.
    \item \textbf{True negatives} are when you predict an observation does not belong to a class and it actually does not belong to that class.
    \item \textbf{False positives} occur when you predict an observation belongs to a class when in reality it does not.
    \item \textbf{False negatives} occur when you predict an observation does not belong to a class when in fact it does.
\end{enumerate}

These four outcomes are often plotted on a confusion matrix. The following confusion matrix is an example for the case of binary classification. This matrix should be generated making predictions on the test data and then identifying each prediction as one of the four possible outcomes described above.

\vspace{0.1in}

\begin{table}[H]
\centering
\begin{tabular}{cc|cc|}
\cline{3-4}
\multicolumn{2}{c|}{\multirow{2}{*}{}}                                                                                        & \multicolumn{2}{c|}{\textbf{Actual Values}}             \\ \cline{3-4} 
\multicolumn{2}{c|}{}                                                                                                         & \multicolumn{1}{c|}{\textbf{Yes (1)}} & \textbf{No (0)} \\ \hline
\multicolumn{1}{|c|}{\multirow{2}{*}{\textbf{\begin{tabular}[c]{@{}c@{}}Predicted\\ Values\end{tabular}}}} & \textbf{Yes (1)} & \multicolumn{1}{c|}{\textbf{TP}}      & \textbf{FP}     \\ \cline{2-4} 
\multicolumn{1}{|c|}{}                                                                                     & \textbf{No (0)}  & \multicolumn{1}{c|}{\textbf{FN}}      & \textbf{TN}     \\ \hline
\end{tabular}
\caption{Confusion Matrix}
\label{Confusion Matrix}
\end{table}


The three main metrics used to evaluate a classification model are accuracy, precision, and recall.

Accuracy is defined as the percentage of correct predictions for the test data. It can be calculated easily by dividing the number of correct predictions by the number of total predictions.

\begin{equation}
  Accuracy = \frac{Correct\; Predictions}{All\; Predictions}
\end{equation}

Precision is defined as the fraction of relevant examples (true positives) among all of the examples which were predicted to belong in a certain class.

\begin{equation}
  Precision = \frac{True\; Positives}{True\; Positives + False\; Positives}
\end{equation}

Recall is defined as the fraction of examples which were predicted to belong to a class with respect to all of the examples that truly belong in the class.

\begin{equation}
  Recall = \frac{True\; Positives}{True\; Positives + False\; Negatives}
\end{equation}

We have a semantic segmentation problem. Therefore, we use the following metrics:

\subsection{Pixel Accuracy}
\hspace{\parindent}
Pixel accuracy is perhaps the easiest to understand conceptually. It is the percent of pixels in the input image that are classified correctly.\

We don't rely on this metric because it is susceptible to class-imbalance, which is when the classes are extremely imbalanced, it means that a class or some classes dominate the image, while some other classes make up only a small portion of the image. Unfortunately, class imbalance is prevalent in many real world data sets, so it can’t be ignored.\

To further illustrate this, if an input image was 100\% black, the output prediction would be above 90\% accurate, which is a totally false prediction as presented in the figure below.

\vspace{0.1in}

\begin{figure}[H]
\centering
  \vspace{-0.1in}
    \centerline{\includegraphics[width = 3.4in]{../images/class_imbalance.png}}
    \caption{Example class imbalance}
\end{figure}

\subsection{IOU}
\hspace{\parindent}
The Jaccard Index or Intersection Over Union, also known as the Jaccard similarity coefficient, is a statistic (metric) used for gauging the similarity and diversity of sample sets. It was developed by Grove Karl Gilbert in 1884 as his ratio of verification (v),\textsuperscript{\cite{murphy1996finley}} and now is frequently referred to as the Critical Success Index in meteorology. It was later developed independently by Paul Jaccard, originally giving the French name 'Coefficient de Communauté'. \textsuperscript{\cite{jaccard1912distribution}} The Jaccard coefficient measures similarity between finite sample sets, and is defined as the size of the intersection divided by the size of the union of the sample sets, here is the formula:

\begin{equation}
    J(A,B) = \frac{Area\; of\; Overlap}{Area\; of\; Union} = \frac{|A \cap B|}{|A \cup B|}
\end{equation}

Here is an example of using IOU on a stop sign, where the green bounding box is the ground truth (the right prediction) and the red bounding box is what the model predicted.

\vspace{0.1in}

\begin{figure}[H]
\centering
  \vspace{-0.1in}
    \centerline{\fbox{\includegraphics[width = 3.5in]{../images/exampleIOU.jpg}}}
    \caption{Example of IOU applied on a stop sign image}
\end{figure}

\subsection{Dice}
\hspace{\parindent}
The Sørensen–Dice coefficient is a statistic used to measure the similarity of two samples, It was developed by the botanists (scientific study of plants) Thorvald Sørensen and Lee Raymond Dice, who published in 1948 and 1945 respectively.\\
Sørensen's original formula was intended to be applied to discrete data. Given two sets, X and Y, it is defined as :
\begin{equation}
    DSC(X, Y) = \frac{2 | X \cap Y |}{| X | + | Y |}
\end{equation}
where |X| and |Y| are the cardinalities of the two sets . The Sørensen index equals twice the number of elements common to both sets divided by the sum of the number of elements in each set as we can see in fig \ref{fig:DSC_EX}. 
When applied to Boolean data, using the definition of true positive (TP), false positive (FP), and false negative (FN), it can be written as :
\begin{equation}
    DSC = \frac{2 TP}{2 TP + FN + FP}
\end{equation}

\begin{figure}[H]
\centering
  \vspace{-0.1in}
    \centerline{\includegraphics[width = 4.8in]{../images/DSC.png}}
    \caption{Example explaining Dice coefficient}
    \label{fig:DSC_EX}
\end{figure}

This coefficient is not very different in form from the Jaccard index (IOU). In fact, both are equivalent in the sense that given a value for the Sørensen–Dice coefficient $S$ , one can calculate the respective Jaccard index value $J$ and vice versa, using the equations : 
\begin{equation}
    J = \frac{DSC}{( 2 − DSC )}
\end{equation}

And 

\begin{equation}
    DSC = \frac{2 J }{( 1 + J )}
\end{equation}

The function ranges between zero and one, like Jaccard. the corresponding loss function:
\begin{equation}
    DSC\_LOSS = 1 - \frac{2 TP}{2 TP + FN + FP}
\end{equation}

\subsection{Tversky}
\hspace{\parindent}
The Tversky index, named after Amos Tversky, is an asymmetric similarity measure on sets that compares a variant to a prototype. The Tversky index can be seen as a generalization of the Sørensen–Dice coefficient and the Dice coefficient (aka Jaccard index).
For sets X and Y the Tversky index is a number between 0 and 1 given by :
\begin{equation}
    S(X, Y) = \frac{| X \cap Y |}{| X \cap Y | + \alpha|X ∖ Y| + \beta | Y ∖ X|}
\end{equation}
Here, $X ∖ Y$ denotes the relative complement of Y in X. 
Further, $\alpha , \beta \geq 0$ are parameters of the Tversky index. Setting $\alpha = \beta = 1$ produces the Jaccard coefficient; setting $\alpha = \beta = 0.5$ produces the Sørensen–Dice coefficient. 

The function ranges between zero and one, the corresponding loss function:

\begin{equation}
    S\_LOSS(X, Y) = 1 - \frac{| X \cap Y |}{| X \cap Y | + \alpha|X ∖ Y| + \beta | Y ∖ X|}
\end{equation}

\subsection{Cross-entropy}
\hspace{\parindent}
Cross-entropy loss, or log loss, measures the performance of a classification model whose output is a probability value between 0 and 1. Cross-entropy loss increases as the predicted probability diverges from the actual label. So predicting a probability of .012 when the actual observation label is 1 would be bad and result in a high loss value. A perfect model would have a log loss of 0.

The cross-entropy for a single example in a binary classification task can be stated by unrolling the sum operation as follows:

\begin{equation}
    H(X, Y) = - (X(class0) * log(Y(class0)) + X(class1) * log(Y(class1)))
\end{equation}

\subsection{Mean Squared Error}
\hspace{\parindent}
Mean squared error (MSE) is simply defined as the average of squared differences between the predicted output and the true output. Squared error is commonly used because it is agnostic to whether the prediction was too high or too low, it just reports that the prediction was incorrect.

This is the Mean Squared Error formula:

\begin{equation}
    MSE = \frac{1}{n} \sum_{i=1}^{n} (Y_{i} - \hat{Y}_{i})^{2}
\end{equation}

We used this loss function in both our DO-U-Net and DO-SegNet models for predicting red and white blood cells.\

There is however a slight problem, the Mean Squared Error has the disadvantage of heavily weighting outliers.[11] This is a result of the squaring of each term, which effectively weights large errors more heavily than small ones.

\section{Conclusion}
\vspace{0.2in}
\hspace{\parindent}
In this chapter we presented the components of our method from segmentation models and their metrics, loss functions and data augmentation. We also explained the counting methods that we are using in our solution.

\newpage


\newpage

\vspace*{\fill}
\begin{center}
    {\color{Black} \rule{\linewidth}{1.2mm} }\\
\vspace{0.25in}
    {\centering\fontsize{30}{40}{\bfseries{\color{Black}{\scshape{Chapter IV : }}}}}
\vspace{0.35in}\\
    {\color{Black} \rule{\linewidth}{1.2mm} }
\end{center}
\vspace*{\fill}
\addcontentsline{toc}{chapter}{\color{Black}{Chapter IV : }}
\setcounter{section}{0}

\newpage

\section{Introduction}
\vspace{0.2in}
\hspace*{0.16in}
In this chapter, we will present artificial intelligence with all of its branches, and then dive deeper into convolutional neural networks (CNNs) and their layers. Finally, we will present some CNN architectures and image processing methods.

\section{Artificial Intelligence}
\subsection{Definition}
It is the science and engineering of making intelligent machines, especially intelligent computer programs. It is related to the similar task of using computers to understand human intelligence, but AI does not have to confine itself to methods that are biologically observable. \textsuperscript{\cite{mccarthy2004artificial}}

\vspace{0.2in}

\begin{figure}[h]
\centering
  \vspace{-0.1in}
    \centerline{\includegraphics[width = 4in, height = 2.2in]{../images/artificial-intelligence.png}}
    \caption{Artificial Intelligence}
\end{figure}

Artificial intelligence algorithms can be categorised into these four main types:

\begin{itemize}
    \item \textbf{Supervised learning:}
        Supervised learning, as the name indicates, has the presence of a supervisor as a teacher. Basically supervised learning is when we teach or train the machine using data that is well labeled. Which means some data is already tagged with the correct answer. After that, the machine is provided with a new set of examples (data) so that the supervised learning algorithm analyses the training data (set of training examples) and produces a correct outcome from labeled data. \textsuperscript{\cite{AITypes-GeeksForGeeks}}

    \item \textbf{Unsupervised learning:}
        Unsupervised learning is the training of a machine using information that is neither classified nor labeled and allowing the algorithm to act on that information without guidance. Here the task of the machine is to group unsorted information according to similarities, patterns, and differences without any prior training of data. \textsuperscript{\cite{AITypes-GeeksForGeeks}}

    \vspace{0.2in}

    \begin{table}[H]
\centering
\resizebox{\textwidth}{!}{%
\begin{tabular}{|
>{\columncolor[HTML]{FFFFFF}}c |
>{\columncolor[HTML]{FFFFFF}}c |
>{\columncolor[HTML]{FFFFFF}}c 
>{\columncolor[HTML]{FFFFFF}}l 
>{\columncolor[HTML]{FFFFFF}}l |}
\hline
\textbf{Parameters}               & \textbf{Supervised machine learning} & \multicolumn{3}{c|}{\cellcolor[HTML]{FFFFFF}\textbf{Unsupervised machine learning}} \\ \hline
\textbf{Input Data} &
  \begin{tabular}[c]{@{}c@{}}Algorithms are trained\\ using labeled data.\end{tabular} &
  \multicolumn{3}{c|}{\cellcolor[HTML]{FFFFFF}\begin{tabular}[c]{@{}c@{}}Algorithms are used against\\ data that is not labeled\end{tabular}} \\ \hline
\textbf{Computational Complexity} & Simpler method                       & \multicolumn{3}{c|}{\cellcolor[HTML]{FFFFFF}Computationally complex}                \\ \hline
\textbf{Accuracy}                 & Highly accurate                      & \multicolumn{3}{c|}{\cellcolor[HTML]{FFFFFF}Less accurate}                          \\ \hline
\end{tabular}%
}
\caption{Table that presents diffrence between supervised and unsupervised machine learning \textsuperscript{\cite{AITypes-GeeksForGeeks}}}
\label{Table that presents diffrence between supervised and unsupervised machine learning}
\end{table}


    \item \textbf{Semi Supervised learning:}
        Semi-supervised learning sits somewhere between Supervised and Unsupervised learning algorithms. It employs a mix of labeled and unlabeled datasets. It works with data that has only a few labels; it usually works with unlabeled data. Labels are expensive, yet for corporate purposes, a few labels may suffice.
    \item \textbf{Reinforcement learning:}
        Reinforcement learning is just a machine learning approach that rewards positive behavior while penalizing poor behavior. In general, a reinforcement learning agent is capable of sensing and interpreting its environment, acting, and learning via trial and error. Developers of reinforcement learning propose a way of rewarding desired behaviors and punishing negative behaviors. \textsuperscript{\cite{SSLvsRL-askanydifference}}

    \vspace{0.2in}

    \begin{table}[H]
\centering
\resizebox{\textwidth}{!}{%
\begin{tabular}{|c|c|cll|}
\hline
\rowcolor[HTML]{FFFFFF} 
\textbf{Parameters}               & \textbf{Semi-Supervised Learning} & \multicolumn{3}{c|}{\cellcolor[HTML]{FFFFFF}\textbf{Reinforcement Learning}} \\ \hline
\rowcolor[HTML]{FFFFFF} 
\textbf{Definition} &
  \begin{tabular}[c]{@{}c@{}}Uses a small amount of labeled\\ data bolstering a larger\\ set of unlabeled data\end{tabular} &
  \multicolumn{3}{c|}{\cellcolor[HTML]{FFFFFF}An algorithm witha reward system} \\ \hline
\rowcolor[HTML]{FFFFFF} 
\textbf{Aim} &
  \begin{tabular}[c]{@{}c@{}}To counter the disadvantages of\\ supervised and unsupervised\\ learning.\end{tabular} &
  \multicolumn{3}{c|}{\cellcolor[HTML]{FFFFFF}\begin{tabular}[c]{@{}c@{}}To learn a series\\ of action\end{tabular}} \\ \hline
\rowcolor[HTML]{FFFFFF} 
\textbf{Interaction of the agent} & Doesn’t interact                  & \multicolumn{3}{c|}{\cellcolor[HTML]{FFFFFF}Interacts}                       \\ \hline
\textbf{Practical application} &
  \begin{tabular}[c]{@{}c@{}}Speech analysis, internet\\ content classification\end{tabular} &
  \multicolumn{3}{c|}{\begin{tabular}[c]{@{}c@{}}Trajectory optimization, motion\\ planning\end{tabular}} \\ \hline
\textbf{Labels}                   & It has labels.                    & \multicolumn{3}{c|}{It doesn’t have labels.}                                 \\ \hline
\end{tabular}%
}
\caption{Comparison Table Between Semi-Supervised and Reinforcement Learning \textsuperscript{\cite{SSLvsRL-askanydifference}}}
\label{Comparison Table Between Semi-Supervised and Reinforcement Learning}
\end{table}

\end{itemize}

\subsection{Machine Learning}
Machine learning is a branch of artificial intelligence (AI) and computer science which focuses on the use of data and algorithms to imitate the way that humans learn, gradually improving its accuracy. \textsuperscript{\cite{ML-IBM}}

\begin{itemize}
  \item \textbf{Linear Regression:}
      Linear Regression is a machine learning algorithm based on supervised learning. It performs a regression task. Regression models a target prediction value based on independent variables. It is mostly used for finding out the relationship between variables and forecasting. Different regression models differ based on – the kind of relationship between dependent and independent variables they are considering, and the number of independent variables getting used. \textsuperscript{\cite{LR-GeeksForGeeks}}
  \item \textbf{Support Vector Machines:}
      Support Vector Machine (SVM) is a computer algorithm that learns by example to assign labels to objects. \textsuperscript{\cite{boser1992training}} For instance, an SVM can learn to recognize fraudulent credit card activity by examining hundreds or thousands of fraudulent and nonfraudulent credit card activity reports. Alternatively, an SVM can learn to recognize handwritten digits by examining a large collection of scanned images of handwritten zeroes, ones and so fourth. \textsuperscript{\cite{noble2006support}}
  \item \textbf{Gradient Descent:}
      Gradient descent is an optimization algorithm used to minimize some function by iteratively moving in the direction of steepest descent as defined by the negative of the gradient. In machine learning, we use gradient descent to update the parameters of our model. Parameters refer to coefficients in Linear Regression and weights in neural networks. \textsuperscript{\cite{DG-ml-cheatsheet}}
\end{itemize}

\subsection{Deep Learning}
\subsubsection{Neural Networks}
Artificial neural networks (ANNs) are comprised of a node layers, containing an input layer, one or more hidden layers, and an output layer. Each node, or artificial neuron, connects to another and has an associated weight and threshold. If the output of any individual node is above the specified threshold value, that node is activated, sending data to the next layer of the network. Otherwise, no data is passed along to the next layer of the network.


\begin{figure}[H]
\centering
\includegraphics[width=\linewidth]{../images/neural-network-diagram.png}
\caption{Architecture of Neural Network}
\label{fig:NN}
\end{figure}

Neural networks rely on training data to learn and improve their accuracy over time. However, once these learning algorithms are fine-tuned for accuracy, they are powerful tools in computer science and artificial intelligence, allowing us to classify and cluster data at a high velocity. Tasks in speech recognition or image recognition can take minutes versus hours when compared to the manual identification by human experts. One of the most well-known neural networks is Google’s search algorithm.

if we dive into the details, we can consider that each node has it's linear regression model, composed of input data, weights, a bias (or threshold), and an output. The formula would look something like equation \ref{eq:NN-Node-Activation}:

%\begin{figure}[H]
%\centering
%\includegraphics[width=\linewidth]{../images/NN-Node-equation.png}
%\caption{Architecture of Neural Network}
%\label{fig:NN}
%\end{figure}

\begin{equation}
    \sum_{i=1}^{m} W_{i}X_{i} + bias = W_{1}X_{1} + W_{2}X_{2} + W_{3}X_{3} + bias
    \label{eq:NN-Node-Activation}
\end{equation}

\begin{equation}
    output = f(x) = 
    \begin{cases}
        1 & if \sum_{i=1}^{m} W_{1}X_{1} + b \leq 0 \\
        0 & if \sum_{i=1}^{m} W_{1}X_{1} + b < 0
    \end{cases}
    \label{eq:NN-Node-Activation2}
\end{equation}

Once an input layer is determined, weights are assigned. These weights help determine the importance of any given variable, with larger ones contributing more significantly to the output compared to other inputs. All inputs are then multiplied by their respective weights and then summed (similar to eaquation \ref{eq:NN-Node-Activation}). Afterward, the output is passed through an activation function as we can see in equation \ref{eq:NN-Node-Activation2}, which determines the output. If that output exceeds a given threshold, it “fires” (or activates) the node, passing data to the next layer in the network. This results in the output of one node becoming in the input of the next node. This process of passing data from one layer to the next layer defines this neural network as a feedforward network.

\subsubsection{Convolutional Neural Networks}
CNNs or ConvNets are among the most successful and widely used architectures in the deep learning community, especially for computer vision tasks. CNNs were initially proposed by Fukushima in his seminal paper on the “Neocognitron” \textsuperscript{\cite{fukushima_Neocognitron}}.

\begin{figure}[H]
\centering
\includegraphics[width=\linewidth]{../images/CNN.png}
\caption{Architecture of convolutional neural networks. From \textsuperscript{\cite{minaee2021image}}}
\label{fig:CNN}
\end{figure}

Convolutional neural networks are distinguished from other neural networks by their superior performance with image, speech, or audio signal inputs. They have three main types of layers, which are:

\begin{itemize}
    \item Convolutional layer
    \item Pooling layer
    \item Fully-connected (FC) layer
\end{itemize}

The convolutional layer is the first layer of a convolutional network. While convolutional layers can chained by additional convolutional layers or pooling layers, the fully-connected layer is the final layer. With each layer, the CNN increases in its complexity, identifying greater portions of the image. Earlier layers focus on simple features, such as colors and edges. As the image data progresses through the layers of the CNN, it starts to recognize larger elements or shapes of the object until it finally identifies the intended object. 
    
    
\begin{enumerate}
    \item \textbf{Convolutional layer} : \\
        The convolutional layer is the core building block of a CNN, and it is where the majority of computation occurs. It requires a few components, which are input data, a filter, and will output a feature map. Let’s assume that the input will be a color image, which is made up of a matrix of pixels in 3D. This means that the input will have three dimensions—a height, width, and depth—which correspond to RGB in an image. We also have a feature detector, also known as a kernel or a filter, which will move across the receptive fields of the image, checking if the feature is present. This process is known as a convolution. \textsuperscript{\cite{CNN-IBM}} \\
        The feature detector is a two-dimensional (2-D) array of weights, which represents part of the image. While they can vary in size, the filter size is typically a 3x3 matrix; this also determines the size of the receptive field. The filter is then applied to an area of the image, and a dot product is calculated between the input pixels and the filter. This dot product is then fed into an output array. Afterwards, the filter shifts by a stride, repeating the process until the kernel has swept across the entire image. The final output from the series of dot products from the input and the filter is known as a feature map, activation map, or a convolved feature.
        \begin{figure}[H]
            \centering
            \includegraphics[width=10cm]{../images/CNN-kernel.png}
            \caption{CNN kernel}
            \label{fig:CNN-kernel}
        \end{figure}
        as we can see in the fig \ref{fig:CNN-kernel}, the kernel will browse all the matrix by shifting it's position. where the weights in the kernel will remain fixed as it moves across the image, which is also known as parameter sharing. Some parameters, like the weight values, adjust during training through the process of backpropagation and gradient descent. However, there are three hyperparameters which affect the volume size of the output that need to be set before the training of the neural network begins. These include:
        \begin{itemize}
            \item \textbf{The number of filters} affects the depth of the output. For example, three distinct filters will give us three different feature maps, creating a depth of three.
            \item \textbf{Stride} is the distance, or number of pixels, that the kernel moves over the input matrix. While stride values of two or greater is rare, a larger stride yields a smaller output.
            \item \textbf{Zero-padding }is usually used when the filters do not fit the input image. This sets all elements that fall outside of the input matrix to zero, producing a larger or equally sized output. There are three types of padding:
            \begin{itemize}
                \item \textbf{Valid padding}: This is also known as no padding. In this case, the last convolution is dropped if dimensions do not align.
                \item \textbf{Same padding}: This padding ensures that the output layer has the same size as the input layer
                \item \textbf{Full padding}: This type of padding increases the size of the output by adding zeros to the border of the input.
            \end{itemize}
        \end{itemize}
        
        \begin{figure}[H]
            \centering
            \includegraphics[width=10cm]{../images/CNN-Feature-Hierarchy.jpg}
            \caption{Feature Hierarchy}
            \label{fig:CNN-Feature-Hierarchy}
        \end{figure}
        
        After each convolution operation, a CNN applies an activation function to the feature map, the most used activation function is the Rectified Linear Unit (ReLU).\\
        As we mentioned earlier, when we chain convolutional layers, the structure of the CNN can become hierarchical as the later layers can see the pixels within the receptive fields of prior layers.  As an example, let’s assume that we’re trying to determine if an image contains a bicycle. You can think of the bicycle as a sum of parts. It is comprised of a frame, handlebars, wheels, pedals, et cetera. Each individual part of the bicycle makes up a lower-level pattern in the neural net, and the combination of its parts represents a higher-level pattern, creating a feature hierarchy within the CNN.
    \item \textbf{Pooling Layer}:\\
        Pooling layers, also known as downsampling, conducts dimensionality reduction, reducing the number of parameters in the input. Similar to the convolutional layer, the pooling operation sweeps a filter across the entire input, but the difference is that this filter does not have any weights. Instead, the kernel applies an aggregation function to the values within the receptive field, populating the output array. There are two main types of pooling:
        \begin{itemize}
            \item \textbf{Max pooling}: As the filter moves across the input, it selects the pixel with the maximum value to send to the output array. As an aside, this approach tends to be used more often compared to average pooling.
            \item \textbf{Average pooling}: As the filter moves across the input, it calculates the average value within the receptive field to send to the output array.
        \end{itemize}
        
        While a lot of information is lost in the pooling layer, it also has a number of benefits to the CNN. They help to reduce complexity, improve efficiency, and limit risk of overfitting.
    
    \item \textbf{Fully-Connected Layer}:\\
        The name of the fully-connected layer aptly describes itself. The pixel values of the input image are not directly connected to the output layer in partially connected layers. However, in the fully-connected layer, each node in the output layer connects directly to a node in the previous layer.

        This layer performs the task of classification based on the features extracted through the previous layers and their different filters. While convolutional and pooling layers tend to use ReLu functions or other activation functions, FC layers usually leverage a softmax or sigmoid activation function to classify inputs appropriately, producing a probability from 0 to 1.
        
\end{enumerate}


\section{CNN architectures}
\vspace{0.2in}
\hspace*{0.16in}

In this section we are going to present some CNN architectures, CNN architectures are preDesigned models targeting a spcial task like image segmentation, feature extraction, classification ..., all the architectures can be an encoder or decoder and it can be both.
\begin{itemize}
    \item \textbf{encoder:} also called contracting path which consist of the downsampling of the feature map by using multiple convolutional and max pooling layers and it differs between model architectures.
    \item \textbf{decoder:} also called expansive path wich consist of an upsampling of the feature map by using up-convolution (which concatenates the feature map from the upper parallel encoder layers)
\end{itemize}

\begin{figure}[h]
    \centering
      \vspace{-0.1in}
        \centerline{\includegraphics[width = 6in, height = 2.2in]{../images/timeline-of-segmentation-algorithms.png}}
        \caption{Timeline of segmentation algorithms}
    \end{figure}


\subsection{VGG}
VGG stands for Visual Geometry Group; it is a standard deep Convolutional Neural Network (CNN) architecture with multiple layers. The “deep” refers to the number of layers with VGG-16 or VGG-19 consisting of 16 and 19 convolutional layers.\\

The VGG architecture is the basis of ground-breaking object recognition models. Developed as a deep neural network, the VGGNet also surpasses baselines on many tasks and datasets beyond ImageNet. Moreover, it is now still one of the most popular image recognition architectures. 

\subsubsection{VGG16}
The VGG model, or VGGNet, that supports 16 layers is also referred to as VGG16, which is a convolutional neural network model proposed by A. Zisserman and K. Simonyan from the University of Oxford in 2014. These researchers published their model in the research paper titled, “Very Deep Convolutional Networks for Large-Scale Image Recognition.” \textsuperscript{\cite{simonyan2014very}} 

The VGG16 model achieves almost 92.7\% top-5 test accuracy in ImageNet. ImageNet is a dataset consisting of more than 14 million images belonging to nearly 1000 classes. Moreover, it was one of the most popular models submitted to ILSVRC-2014. It replaces the large kernel-sized filters with several 3×3 kernel-sized filters one after the other, thereby making significant improvements over AlexNet. The VGG16 model was trained using Nvidia Titan Black GPUs for multiple weeks.

As mentioned above, the VGGNet-16 supports 16 layers and can classify images into 1000 object categories, including keyboard, animals, pencil, mouse, etc. Additionally, the model has an image input size of 224-by-224.

\subsubsection{VGG19}
The concept of the VGG19 model (also VGGNet-19) is the same as the VGG16 except that it supports 19 layers. The “16” and “19” stand for the number of weight layers in the model (convolutional layers). This means that VGG19 has three more convolutional layers than VGG16. 

\subsubsection{VGG Architecture}
VGGNets are based on the most essential features of convolutional neural networks (CNN).
The VGG network is constructed with very small convolutional filters. The VGG-16 consists of 13 convolutional layers and three fully connected layers.
To sum it all up, VGG architecture consists of:

\begin{itemize}
    \item \textbf{Input:} The VGGNet takes in an image input size of 224×224. For the ImageNet competition, the creators of the model cropped out the center 224×224 patch in each image to keep the input size of the image consistent.
    \item \textbf{Convolutional Layers:} VGG’s convolutional layers leverage a minimal receptive field, i.e., 3×3, the smallest possible size that still captures up/down and left/right. Moreover, there are also 1×1 convolution filters acting as a linear transformation of the input. This is followed by a ReLU unit, which is a huge innovation from AlexNet that reduces training time. ReLU stands for rectified linear unit activation function; it is a piecewise linear function that will output the input if positive; otherwise, the output is zero. The convolution stride is fixed at 1 pixel to keep the spatial resolution preserved after convolution (stride is the number of pixel shifts over the input matrix).
    \item \textbf{Hidden Layers:} All the hidden layers in the VGG network use ReLU. VGG does not usually leverage Local Response Normalization (LRN) as it increases memory consumption and training time. Moreover, it makes no improvements to overall accuracy.
    \item \textbf{Fully-Connected Layers:} The VGGNet has three fully connected layers. Out of the three layers, the first two have 4096 channels each, and the third has 1000 channels, 1 for each class.
\end{itemize}

\subsubsection{VGG16 architecture}
The number 16 in the name VGG refers to the fact that it is 16 layers deep neural network (VGGnet). This means that VGG16 is a pretty extensive network and has a total of around 138 million parameters. Even according to modern standards, it is a huge network. However, VGGNet16 architecture’s simplicity is what makes the network more appealing. Just by looking at its architecture, it can be said that it is quite uniform.

There are a few convolution layers followed by a pooling layer that reduces the height and the width. If we look at the number of filters that we can use, around 64 filters are available that we can double to about 128 and then to 256 filters. In the last layers, we can use 512 filters.

\begin{figure}[h]
\centering
    \centerline{\includegraphics[width = 4.2in, height = 2.4in]{../images/VGG16-architecture.png}}
    \caption{VGG16 architecture}
\end{figure}

\subsubsection{Performance of VGG Models}
VGG16 highly surpasses the previous versions of models in the ILSVRC-2012 and ILSVRC-2013 competitions. Moreover, the VGG16 result is competing for the classification task winner (GoogLeNet with 6.7\% error) and considerably outperforms the ILSVRC-2013 winning submission Clarifai. It obtained 11.2\% with external training data and around 11.7\% without it. In terms of the single-net performance, the VGGNet-16 model achieves the best result with about 7.0\% test error, thereby surpassing a single GoogLeNet by around 0.9\%. \textsuperscript{\cite{VGG-Gaudenz_Boesch}}

\subsection{U-Net}
U-Net is a CNN architecture built for biomedical image segmentation in 2015 \textsuperscript{\cite{ronneberger2015u}}. It consists of a contracting path (encoder) and an expansive path (decoder). The encoder follows the typical architecture of a convolutional network. It consists of the repeated application of two 3x3 convolutions (unpadded convolutions), each followed by a rectified linear unit (ReLU) and a 2x2 max pooling operation with stride 2 for downsampling. At each downsampling step we double the number of feature channels. Every step in the expansive path consists of an upsampling of the feature map followed by a 2x2 convolution (“up-convolution”) that halves the number of feature channels, a concatenation with the correspondingly cropped feature map from the contracting path, and two 3x3 convolutions, each followed by a ReLU. The cropping is necessary due to the loss of border pixels in every convolution. At the final layer a 1x1 convolution is used to map each 64-component feature vector to the desired number of classes. In total the network has 23 convolutional layers \textsuperscript{\cite{paperswithcode-U-Net}}.

\begin{figure}[H]
\centering
\includegraphics[width=\columnwidth]{../images/u-net-architecture.png}
\caption{U-Net Architecture. From \textsuperscript{\cite{ronneberger2015u}}}
\label{fig:CNN}
\end{figure}

\subsubsection{Performance}
The u-net is fast and precise model for segmentation of images. Up to now it has outperformed the prior best method (a sliding-window convolutional network) on the ISBI challenge for segmentation of neuronal structures in electron microscopic stacks. It has won the Grand Challenge for Computer-Automated Detection of Caries in Bitewing Radiography at ISBI 2015, and it has won the Cell Tracking Challenge at ISBI 2015 on the two most challenging transmitted light microscopy categories (Phase contrast and DIC microscopy) by a large margin.

\subsection{SegNet}

SegNet is a semantic segmentation model designed in 2017. This core trainable segmentation architecture consists of an encoder network, a corresponding decoder network followed by a pixel-wise classification layer. The architecture of the encoder network is topologically identical to the 13 convolutional layers in the VGG16 network. The role of the decoder network is to map the low resolution encoder feature maps to full input resolution feature maps for pixel-wise classification. The novelty of SegNet lies is in the manner in which the decoder upsamples its lower resolution input feature maps. Specifically, the decoder uses pooling indices computed in the max-pooling step of the corresponding encoder to perform non-linear upsampling. \textsuperscript{\cite{badrinarayanan2017segnet}}

\begin{figure}[h]
\centering
  \vspace{-0.1in}
    \centerline{\includegraphics[width = 4in, height = 2.2in]{../images/segnet.png}}
    \caption{SegNet architecture}
\end{figure}

\section{Image Processing Methods}
\vspace{0.2in}
\hspace*{0.16in}

\subsection{Watershed}

\subsection{Discrete cosine transform}
A discrete cosine transform (DCT) expresses a finite sequence of data points in terms of a sum of cosine functions oscillating at different frequencies. The DCT, first proposed by Nasir Ahmed in 1972 \textsuperscript{\cite{ahmed1974discrete}}, is a widely used transformation technique in signal processing and data compression. It is used in most digital media, including digital images, digital video, digital audio, digital television, digital radio, and speech coding. DCTs are also important to numerous other applications in science and engineering, such as digital signal processing, telecommunication devices, reducing network bandwidth usage, and spectral methods for the numerical solution of partial differential equations.
the equation for the discrete cosine transform is represented as:

\begin{equation}
    c(u,v)=\alpha (u)\alpha (v)\sum_{u=0}^{N-1}\sum_{v=0}^{N-1} f(x,y) cos {\begin{bmatrix} {{\pi (2x + 1)u}\over{2N}} \end{bmatrix}} cos {\begin{bmatrix} {{\pi (2y + 1)v}\over{2N}} \end{bmatrix}} ,
    \label{eq:Discrete cosine transform equation}
\end{equation}

\begin{figure}[H]
\centering
    \centerline{\includegraphics[width = 4.6in, height = 4.2in]{../images/inverse-DCT-of-trees.png}}
    \caption{Inverse DCT of Trees; (a) DCT(100\%); (b) DCT(75\%); (c) DCT(50\%); (d) DCT(25\%).}
\end{figure}



\newpage


\newpage

\vspace*{\fill}
\begin{center}
    {\color{Black} \rule{\linewidth}{1.2mm} }\\
\vspace{0.25in}
{\centering\fontsize{30}{40}{\bfseries{\color{Black}{\scshape{Chapter V : Methodology}}}}}
\vspace{0.35in}\\
    {\color{Black} \rule{\linewidth}{1.2mm} }
\end{center}
\vspace*{\fill}
\addcontentsline{toc}{chapter}{\color{Black}{Chapter V : Methodology}}
\setcounter{section}{0}

\newpage

\section{Introduction}
\vspace{0.2in}
\hspace*{0.16in}
In our case study, and from multiple articles, we can see that U-Net and Segnet models are dominating the field of cell segmentation.
In this chapter, we test out both U-Net and Segnet models, and analyse the results by comparing the two architectures.
We will also explore different machine learning algorithms for both preprocessing and postprocessing.

\section{Our approach}
\vspace{0.2in}
\hspace*{0.16in}
From all of the intel we have gathered, and previously read articles, all of cell segmentation (blood cell segmentation in particular) are mostly using U-Net and Segnet archtectures for segmenting blood cell images.
We have implemented both the U-Net and Segnet models.
In the following sections, a brief study analyzing and comparing both models with their perspective results.

\section{U-Net}
\subsection{Definition}
\lipsum[2-2]

\section{Segnet}
\subsection{Definition}
The SegNet neural network, developed by Alex Kendall, Vijay Badrinarayanan, and Roberto Cipolla, all from the University of Cambridge, is a convolutional neural network used for semantic pixel wise labeling. This problem is more commonly called semantic segmentation. \textsuperscript{\cite{badrinarayanan2017segnet}}

\subsection{Architecture}
SegNet has an encoder network and a corresponding decoder network, followed by a final pixelwise classification layer. This architecture is illustrated in the below figure.
With 13 encoder layers obtained from the VGG16 network, and 13 decoder layers to match the same number of encoder layers. The final decoder output is fed to a multi-class soft-max classifier to produce class probabilities for each pixel independently (pixelwise).

Each encoder in the encoder network performs convolution with a filter bank to produce a set of feature maps. These are then batch normalized. Then an element-wise rectified- linear non-linearity (ReLU) max(0, x) is applied. Following that, max-pooling with a 2x2 window and stride 2 (non-overlapping window) is performed and the resulting output is sub-sampled by a factor of 2. Max-pooling is used to achieve translation invariance over small spatial shifts in the input image.

The appropriate decoder in the decoder network upsamples its input feature map(s) using the memorized max-pooling indices from the corresponding encoder feature map(s). This step pro- duces sparse feature map(s). This SegNet decoding technique is illustrated in the below figure.
These feature maps are then convolved with a trainable decoder filter bank to produce dense feature maps. A batch normalization step is then applied to each of these maps. Note that the decoder corresponding to the first encoder (closest to the input image) produces a multi-channel feature map, although its encoder input has 3 channels (RGB).
This is unlike the other decoders in the network which produces feature maps with the same number of size and channels as their encoder inputs. The high dimensional feature representation at the output of the final decoder is fed to a trainable soft-max classifier.
This soft-max classifies each pixel independently. The output of the soft-max classifier is a K channel image of probabilities where K is the number of classes. The predicted segmentation corresponds to the class with maximum probability at each pixel. \textsuperscript{\cite{badrinarayanan2017segnet}}

\vspace{0.2in}

\begin{figure}[h]
\centering
  \vspace{-0.1in}
    \centerline{\includegraphics[width = 4in, height = 2.2in]{../images/segnet.png}}
    \caption{SegNet architecture}
\end{figure}

\subsection{Dataset}
For the segnet model, we have decided to test th ALL-IDB1 dataset, which contains 108 blood cell images, of which 10 images with their perspective masks and edge masks were chosen for red blood cell training and 3 as a test dataset, for white blood cells 73 images with their masks, and 33 as a test dataset.
For platelets, we used 71 for training and 31 as a test dataset.
Only red blood cells have edge masks, because we need to get rid of overlapped cells, white blood cells and platelets dont need the edge masks, using masks only can retrieve all the necessary features, because both white blood cells and platelets rarely overlap.

\subsection{Dataset augmentation}



\newpage

\vspace*{\fill}
\begin{center}
    {\color{Black} \rule{\linewidth}{1.2mm} }\\
\vspace{0.25in}
{\centering\fontsize{30}{40}{\bfseries{\color{Black}{\scshape{Chapter VI : Experiments And Results}}}}}
\vspace{0.35in}\\
    {\color{Black} \rule{\linewidth}{1.2mm} }
\end{center}
\vspace*{\fill}
\addcontentsline{toc}{chapter}{\color{Black}{Chapter VI : Experiments And Results}}
\setcounter{section}{0}

\newpage

\section{Introduction}
\vspace{0.2in}
\hspace*{0.16in}
%TODO explain what we tested with steps 

\section{Used Tools}
\vspace{0.2in}
\hspace*{0.16in}


\section{Results}

\subsection{Red Blood Cells}
\subsubsection{DO-UNET}
%TODO we will have a table which has accuracy and metrics for example 10 images and their real and predicted count with the 3 methods
%example of table columns
% Image Results(Metrics IOU ...) Real-Count CHT CCL Watershed

\subsection{DO-SegNet}
%TODO we will do the same thing as DO-UNET and at the end we draw a resume table wich will compare between both/ or we can do both in the same table 

\subsection{White Blood Cells}
%TODO Same
\subsubsection{DO-UNET}
\subsubsection{DO-SegNET}


\subsection{Platlets}
%TODO Same
\subsubsection{DO-UNET}
\subsubsection{DO-SegNET}

%TODO at the end we draw some tables that will contain the 3 

\subsection{}
\vspace{0.2in}
\hspace*{0.16in}

\newpage

\vspace*{0.2in}

\thispagestyle{empty}

\begin{center}
    {\color{Black} \rule{4in}{1.4mm} }\\
    \vspace{0.1in}
    \scshape{\fontsize{34}{46}{\bfseries{\color{Black}{Conclusion}}}}
    \\
    \vspace{0.6in}
\end{center}
\cftaddtitleline{toc}{part}{\vspace{-0.12in}\color{Black}{Conclusion}}{}
\begin{changemargin}{0.9cm}{0.9cm}
\hspace*{0.16in}
\end{changemargin}

\newpage

\bibliography{main}

\end{document}
